\documentclass{beamer}
\graphicspath{{Figures/}}

\usepackage{amsmath, amssymb, graphicx, float}
\usepackage{ragged2e}

%\usepackage{AnnArbor}
%\usetheme{Berkeley}
%\usetheme{Berlin}
\usetheme{boxes}
%\usepackage{CambridgeUS}
%\usepackage{Copenhagen}
%\usetheme{Dresden}
%\usetheme{default}
%\usetheme{Frankfurt}
%\usetheme{Hannover}
%\usetheme{Madrid}
%\usetheme{Rochester}
%\usetheme{Pittsburgh}
%\usetheme{Rochester}
%\usepackage{Singapore}
%\usetheme{Warsaw}

\usepackage{pgfpages}
\setbeameroption{show notes on second screen=right}
\setbeamerfont{title}{family=\rm}
\usefonttheme{serif}
%\setmainfont{Liberation Serif}

%\title{{\normalsize Large-scale structure of complex networks (Part 1)}}
\title{{Large-scale structure of complex networks (Part 1)}}
\author{\small Snehal M. Shekatkar}
\institute{Centre for modeling and simulation,\\  S.P. Pune University, Pune}
\date{}

\begin{document}

%--------------------------------------
\begin{frame}
    \frametitle{}
    \maketitle
\end{frame}
%--------------------------------------
%-------------------------
\begin{frame}
    \frametitle{}
    \centering
    \includegraphics[width = 0.6\columnwidth]{networks_book.jpeg}
    \note{Most material is covered in this book}
\end{frame}
%-------------------------
%--------------------------------------
\begin{frame}
    \frametitle{Networks/Graphs}
    \begin{columns}
        \column{0.45\linewidth}
            \centering
            \includegraphics[width=\columnwidth]{lesmis.pdf}
        \column{0.58\linewidth}
            \begin{itemize}
                \setlength\itemsep{2em}
                \item{Points connected by lines}
                \item{Points: nodes/vertices/actors}
                \item{Lines: links/edges/ties}
                \item{Degree: number of neighbors}
            \end{itemize}
    \end{columns}
\end{frame}
%--------------------------------------
%--------------------------------------
\begin{frame}
    \frametitle{Real-world networks}
    \begin{itemize}
        \setlength\itemsep{1em}
            \item{{\bf Social}: Facebook, Friendships, Scientific collaborations}
            \item{{\bf Biological}: Human brain, Metabolic reactions }
            \item{{\bf Technological}: Internet, World-Wide-Web}
            \item{{\bf Transport}: Airports-Air routes, Cities-Highways}
    \end{itemize}
\end{frame}
%--------------------------------------
%--------------------------------------
%\begin{frame}
%    \frametitle{Real-world networks}
%    \begin{figure}
%        \begin{center}
%            \includegraphics[width=0.8\columnwidth]{airports_network_India.pdf}
%        \end{center}
%    \end{figure}
%\end{frame}
%--------------------------------------
%--------------------------------------
\begin{frame}
    \frametitle{Two-questions}
        \begin{center}
            {\huge What are the nodes?}\\
            \vspace{5em}
            {\huge What are the links?}
        \end{center}
\end{frame}
%--------------------------------------
%--------------------------------------
\begin{frame}
    \frametitle{Is this a network?}

\begin{figure}
    \begin{center}
        \includegraphics[width=0.6\columnwidth]{galaxy.jpeg}
    \end{center}
\end{figure}

\end{frame}
%--------------------------------------
%--------------------------------------
\begin{frame}
    \frametitle{Is this a network?}
\begin{figure}
    \begin{center}
        \includegraphics[width=0.4\columnwidth]{crystal.jpeg}
    \end{center}
\end{figure}
\end{frame}
%--------------------------------------
%--------------------------------------
\begin{frame}
    \frametitle{Is this a network?}
\begin{figure}
    \begin{center}
        \includegraphics[width=0.7\columnwidth]{cauliflower.jpeg}
    \end{center}
\end{figure}
\end{frame}
%--------------------------------------
%--------------------------------------
\begin{frame}
    \frametitle{When is a network description useful?}
    \begin{itemize}
    \setlength\itemsep{1em}
        \item{Sparse data}
        \item{Lack of regularity}
        \item{Lack of a better model}
    \end{itemize}
\end{frame}
%--------------------------------------
%-------------------------
\begin{frame}
    \frametitle{Adjacency matrix}
    \centering
    For a network with $N$ nodes, consider a $n\times n$ matrix ${\mathbf A}$

$$A_{ij} = \begin{cases}1 &\quad \text{if the nodes $i$ and $j$ are connected}\\0 &\quad \text{otherwise}\end{cases}$$

\note{For undirected networks, A is symmetric}
\end{frame}
%-------------------------
%--------------------------------------
\begin{frame}
    \frametitle{Complex networks}
    \begin{itemize}
    \setlength\itemsep{2em}
       \item{\Large {\bf Complex}: Edge of order and randomness}
        \pause
        \item{\Large {\bf Structure vs Processes}
            \begin{itemize}
            \setlength\itemsep{1em}
                \item{Spreading of epidemics, rumors, ideas}
                \item{Traffic}
                \item{Neuronal dynamics}
            \end{itemize}
        }
        \pause
        \item{\Large \bf Structure is intersting on its own!}
    \end{itemize}
\end{frame}
%--------------------------------------
%--------------------------------------
\begin{frame}
    \frametitle{Simplifications}
    \begin{columns}
        \column{0.7\linewidth}
            \centering
            \includegraphics[width=0.9\columnwidth]{weighted_directed_nonsimple2.pdf}
        \column{0.4\linewidth}
            \centering
            \begin{itemize}
            \setlength\itemsep{1em}
                \item{Simple}
                \item{Undirected}
                \item{Unweighted}
                \item{Static}
            \end{itemize}
    \end{columns}
\end{frame}
%--------------------------------------

%----------------------------------------
\begin{frame}
    \frametitle{Simplifications}
    \centering
    {\Large \bf The largest component}
    \begin{figure}
        \includegraphics[width=0.7\columnwidth]{netscience.pdf}
    \end{figure}
\end{frame}
%----------------------------------------

%--------------------------------------
\begin{frame}
    \frametitle{Large-scale structure of complex networks}
    \begin{itemize}
    \setlength\itemsep{1em}
        \item{{\bf Small-scale structures}: 
            \begin{itemize}
                \item{degree}
                \item{local clustering}
                \item{centrality scores}
            \end{itemize}
}
        \item{{\bf Meso-scale structures}: 
            \begin{itemize}
                \item{motifs}
                \item{vertex similarity}
                \item{rich-club effect}
            \end{itemize}
}
        \item{{\bf Large-scale structures}: 
                \begin{itemize}
                    \item{components and percolation}
                    \item{small-world effect}
                    \item{ranking}
                    \item{{\bf degree distribution}}
                    \item{{\bf assortative mixing}}
                    \item{{\bf community structure}}
                \end{itemize}
}
    \end{itemize}
    
\end{frame}
%--------------------------------------
%--------------------------------------
\begin{frame}
    \frametitle{Degree-distrbution}
    \begin{columns}
        \column{0.5\linewidth}
        \centering
        \includegraphics[width=\columnwidth]{small_graph.pdf}
        \column{0.5\linewidth}
        \centering
        Total $10$ vertices

        $$p_{1} = \frac{3}{10}$$
        $$p_{2} = \frac{2}{10}$$
        $$p_{3} = \frac{2}{10}$$
        $$p_{4} = \frac{2}{10}$$
        $$p_{5} = \frac{1}{10}$$
    \end{columns}
\end{frame}
%--------------------------------------
%--------------------------------------
\begin{frame}
    \frametitle{Degree distribution}
    \begin{columns}
        \column{0.6\linewidth}
        \centering
        \includegraphics[width=\columnwidth]{big_graph.pdf}

        \column{0.6\linewidth}
        \includegraphics[width=\columnwidth]{deg_distri_example.pdf}
        \centering
    \end{columns}
\end{frame}
%--------------------------------------
%--------------------------------------
\begin{frame}
    \frametitle{Metabolic network of the worm C-elegans}
    \begin{columns}
        \column{0.6\linewidth}
        \centering
        \includegraphics[width=\columnwidth]{celegans_metabolic.pdf}
        \column{0.6\linewidth}
        \centering
        \includegraphics[width=\columnwidth]{deg_distri_celegans_metabolic.pdf}
    \end{columns}
\end{frame}
%--------------------------------------
%--------------------------------------
\begin{frame}
    \frametitle{Degree distribution of the real world networks}
    \begin{columns}
        \column{0.6\linewidth}
        \centering
        \includegraphics[width=\columnwidth]{airports_network_India.pdf}

        \column{0.6\linewidth}
        \centering
        \includegraphics[width=\columnwidth,trim=30 0 0 0,clip=true]{deg_distri_india.pdf}
    
    \end{columns}
\end{frame}
%--------------------------------------
%--------------------------------------
\begin{frame}
    \frametitle{Degree distribution of the real-world networks}
    \begin{columns}
        \column{0.6\linewidth}
        \centering
        \includegraphics[width=\columnwidth]{airports_network_global.pdf}
        \column{0.6\linewidth}
        \includegraphics[width=\columnwidth]{deg_distri_global_airport.pdf}
        \centering
    \end{columns}
\end{frame}
%--------------------------------------
%--------------------------------------
\begin{frame}
    \frametitle{Power-laws and scale-free networks}
    \begin{columns}
        \column{0.7\linewidth}
        \centering
        \includegraphics[width=\columnwidth]{deg_distri_global_airport_log.pdf}
        \column{0.4\linewidth}
        \centering
        $$\ln p(k) = -\alpha \ln k + c$$ 
        $$p(k) = Ck^{-\alpha}\quad\forall \ k > k_{\text{min}}$$

    \end{columns}
\end{frame}
%--------------------------------------
%-------------------------
%\begin{frame}
%    \frametitle{Detecting and visualizing power-laws}
%    \centering
%    How do we know that a given distribution is a power-law?
%    \begin{itemize}
%    \setlength\itemsep{1em}
%        \item{Plotting the distribution on a log-log scale}
%    \end{itemize}
%    \includegraphics[width=0.8\columnwidth]{lognormal.pdf}
%    \note{lognormal}
%\end{frame}
%-------------------------
%-------------------------
\begin{frame}
    \frametitle{Detecting and visualizing power-laws}
    \centering
    How do we know that a given distribution is a power-law?
    \begin{itemize}
    \setlength\itemsep{1em}
        \item{Creating a log-log plot}
    \end{itemize}
    \includegraphics[width=0.8\columnwidth]{deg_distri_global_airport_log.pdf}
    \note{More imp question\\.\\log-log: objectively bad way}
\end{frame}
%-------------------------
%-------------------------
\begin{frame}
    \frametitle{Detecting and visualizing power-laws}
    \centering
    Power-law is tricky!
    \includegraphics[width=0.8\columnwidth]{lognormal.pdf}
    \note{lognormal}
\end{frame}
%-------------------------
%-------------------------
\begin{frame}
    \frametitle{Detecting and visualizing power-laws}
    \centering
    \begin{itemize}
    \setlength\itemsep{1em}
        \item{Logarthmic binning: next bin is fixed multiple wider than the previous one}
        \item{Better but still noisy}
    \end{itemize}

\begin{figure}
    \begin{center}
        \includegraphics[width=0.8\columnwidth]{deg_distri_global_airport_log_logbins.pdf}
        \caption{\label{}}
    \end{center}
\end{figure}

    
\end{frame}
%-------------------------
%-------------------------
\begin{frame}
    \frametitle{Detecting and visualizing power-laws}
    \centering
    {\bf Cumulative distribution}
    $$P(k) = \sum\limits_{k^{\prime} = k}^\infty p(k^{\prime})$$

    $$P(k) = C\sum\limits_{k^{\prime} = k}^\infty {k^{\prime}}^{-\alpha} \approx C\int_k^\infty {k^{\prime}}^{-\alpha}dk^{\prime} = \frac{C}{\alpha-1}k^{-(\alpha-1)}$$
\note{$\alpha > 1$ and power-law slowly decreases\\}
    \note{Needs no binning!}
\end{frame}
%-------------------------
%--------------------------------------
%\begin{frame}
%    \frametitle{Detecting and visualizing power-laws}
%    \centering
%    {\bf The global network of airports}
%    
%    \vspace{2em}
%    \begin{columns}
%    \column{0.35\linewidth}
%        \centering
%        {\bf Log-log scale}
%        \includegraphics[width=\columnwidth]{deg_distri_global_airport_log.pdf}
%    \column{0.35\linewidth}
%        \centering
%        {\bf Logarthmic bins}
%        \includegraphics[width=\columnwidth]{deg_distri_global_airport_log_logbins.pdf}
%    \column{0.35\linewidth}
%        \centering
%        {\bf Cumulative }
%        \includegraphics[width=\columnwidth]{deg_distri_global_airport_log_logbins_cumul.pdf}
%    \end{columns}
%\end{frame}
%--------------------------------------
%--------------------------------------
\begin{frame}
    \frametitle{Detecting and visualizing power-laws}
    \centering
    {\bf A portion of the internet \footnote{Taken from the website of Mark Newman}}

    \vspace{2em}
    \begin{columns}
    \column{0.35\linewidth}
        \centering
        {\bf Log-log scale}
        \includegraphics[width=\columnwidth]{internet_loglog_hist.pdf}
    \column{0.35\linewidth}
        \centering
        {\bf Logarthmic bins}
        \includegraphics[width=\columnwidth]{internet_loglog_logbin_hist.pdf}
    \column{0.35\linewidth}
        \centering
        {\bf Cumulative }
        \includegraphics[width=\columnwidth]{internet_loglog_logbin_cum_hist.pdf}
    \end{columns}
\end{frame}
%--------------------------------------
%--------------------------------------
\begin{frame}
    \frametitle{}
    Calculation of the scaling exponent
    
    \centering
    $$\alpha = 1 + \frac{N}{\left[\sum\limits_i\frac{k_i}{k_{\text{min}}-\frac{1}{2}}\right]}$$

    \justifying
    Statistical error on $\alpha$
    $$\sigma = \frac{\alpha-1}{\sqrt{N}}$$
\end{frame}
%--------------------------------------
%--------------------------------------
\begin{frame}
    \frametitle{Assortative mixing}
    \vspace{2em}
    \centering
    {\small Social network of school-children with two races: Black and White}
    \begin{figure}
        \begin{center}
        %\includegraphics[width=0.8\columnwidth,trim=20 200 200 20,clip=true]{assortativity_high_school.pdf}
        \includegraphics[width=0.8\columnwidth,trim=0 0 0 50,clip=true]{assort_network.pdf}
        \end{center}
    \end{figure}
\end{frame}
%--------------------------------------
%--------------------------------------
\begin{frame}
    \frametitle{Assortative mixing}
        \begin{itemize}
        \setlength\itemsep{1em}
            \item{{\bf Social networks:} race, age, physical locations, language, income, educational level}
            \item{{\bf Citation networks:} topics of the study}
            \item{{\bf World Wide Web:} contents of the webpages}
            \item{{\bf Internet:} physical locations}
        \end{itemize}
\end{frame}
%--------------------------------------
%--------------------------------------
\begin{frame}
    \centering
        \vspace{1em}
    {\bf Assortative mixing by enumerative characteristics}
        \begin{itemize}
        \setlength\itemsep{1em}
            \item{Characteristics with a finite set of values} 
            \item{No ordering} 
            \item{Nationality, Gender, Race} 
        \end{itemize}
        \vspace{1em}
    {\bf Assortative mixing by scalar characteristics}
        \begin{itemize}
        \setlength\itemsep{1em}
            \item{Characteristics with a finite or infinite set of values}
            \item{Ordering}
            \item{Age, income, degree}
        \end{itemize}
        \pause
        \vspace{1em}
        The network is {\bf assortative} if a large fraction of the edges fall between vertices of the same type

        \vspace{1em}
        If the opposite is true, the network is called {\bf dissortative}
\end{frame}
%--------------------------------------
%--------------------------------------
\begin{frame}
    \frametitle{Quantification for enumerative characteristics}
        
    \begin{itemize}
    \setlength\itemsep{2em}
        \item{Fraction of edges connecting vertices of the same type?}
        
        \pause
        \item{Fraction of the actual minus the expected number of edges connecting the vertices when connections are made at random\\

        \pause
        \item{Has value $0$ in trivial cases}
}
    \end{itemize}
\end{frame}
%--------------------------------------
%--------------------------------------
\begin{frame}
    \frametitle{Number of edges between the same types}
        \begin{center}
        $c_i$ : the class or type of vertex $i$\\
        \vspace{2em}
        $n_c$ : total number of types
        \vspace{2em}

        Total number of edges between the vertices of the same type:
        $$\sum\limits_{\text{edges}(i,j)}\delta(i,j)=\frac{1}{2}\sum\limits_{i,j}A_{ij}\delta(c_i,c_j)$$ 
        \end{center}
\end{frame}
%--------------------------------------
%--------------------------------------
\begin{frame}
    \frametitle{Expected number of edges between the same types}
    \begin{itemize}
    \setlength\itemsep{1em}
        \item{Half-edges or stubs, degrees preserved}
        \item{For a given stub at vertex $i$, there are $2m-1$ stubs to which it can connect to}
        \item{Probability of connecting vertex $j$ is $\frac{k_j}{2m}$}
        \item{Expected number of edges between $i$ and $j$ is $\frac{k_ik_j}{2m-1}$}
        \item{Expected number of edges between all the pairs of the same type:
            $$\frac{1}{2}\sum\limits_{ij}\frac{k_ik_j}{2m}\delta(c_i,c_j)$$
}
    \end{itemize}
\end{frame}
%--------------------------------------
%--------------------------------------
\begin{frame}
    \frametitle{Modularity}
        $$\frac{1}{2}\sum\limits_{i,j}A_{ij}\delta(c_i,c_j)-\frac{1}{2}\sum\limits_{ij}\frac{k_ik_j}{2m}\delta(c_i,c_j)=\frac{1}{2}\sum\limits_{ij}\left(A_{ij}-\frac{k_ik_j}{2m}\right)\delta(c_i,c_j)$$ 

        \vspace{2em}
        \pause
        $$Q = \frac{1}{2m}\sum\limits_{ij}\left(A_{ij}-\frac{k_ik_j}{2m}\right)\delta(c_i,c_j)$$

        \vspace{2em}
        \centering
        $Q$ is called the {\bf modularity} of the network w.r.t. to $c$

        $$B_{ij}=A_{ij}-\frac{k_ik_j}{2m}$$
\end{frame}
%--------------------------------------
%--------------------------------------
\begin{frame}
    \frametitle{Normalized modularity}
    \centering
    Modularity is not $1$ even for a perfectly mixed network.

    \pause

    $$Q_{\text{max}} = \frac{1}{2m}\left(2m-\sum\limits_{ij}\frac{k_ik_j}{2m}\delta(c_i,c_j)\right)$$

    \pause
    $$Q_{\text{norm}} = \frac{Q}{Q_{\text{max}}}$$
\end{frame}
%--------------------------------------
%--------------------------------------
\begin{frame}
    \frametitle{Quantification for scalar characteristics}
    \centering
    $x_i$ : a scalar value for vertex $i$
    \vspace{1em}

    $$r = \frac{\sum\limits_{ij}(A_{ij}-k_ik_j/2m)x_ix_j}{\sum\limits_{ij}(k_i\delta_{ij}-k_ik_j/2m)x_ix_j}$$
\end{frame}
%--------------------------------------
%--------------------------------------
\begin{frame}
    \frametitle{Degree-correlations/Degree-assortativity}
    
    \begin{itemize}
    \setlength\itemsep{1em}
        \item{Using degree itself as a scalar property of the nodes}
        \item{Degree is the property of the network structure}
        \item{One property (degree) dictating the others (position of the edges)}
    \end{itemize}
\end{frame}
%--------------------------------------
%--------------------------------------
\begin{frame}
    \frametitle{}
    \centering
    \includegraphics[width=\columnwidth]{assortativity_list.png}
\footnote{Newman, M.E.J., Assortative Mixing in Networks, PRL, 89, 20.}
\end{frame}
%--------------------------------------
%--------------------------------------
\begin{frame}
    \frametitle{}
\end{frame}
%--------------------------------------
\end{document}
